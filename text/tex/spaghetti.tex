\begin{figure}
  \centering
  \subfigure{\includegraphics[width=0.45\textwidth]
             {fig/sims/006941/input.png}} \\
  \subfigure{\includegraphics[width=0.45\textwidth]
             {fig/sims/007022/input.png}} \\
  \subfigure{\includegraphics[width=0.45\textwidth]
             {fig/sims/006919/input.png}}
  \caption{Examples of Spaghetti input.  The models from these appear
    later in Figures \ref{fig:6941}, \ref{fig:7022} and \ref{fig:6919}.
  \label{fig:input-spag}}
\end{figure}

\subsection{\spl} \label{sec:SpaghettiLens}

In \spl the user input is an educated guess for the topography of the
arrival-time surface.

A saddle point is recognizable in a contour map because contours do
not loop around it, they tend to approach and then pull away.  There
is one contour ---the saddle-point contour--- that makes an X on the
saddle point itself.  A saddle point contour may not appear on a plot
(and Figure~\ref{fig:arriv} does not show one.  But we can tell from
the other contours where the saddle-point contour would be.

It provides a web based graphical user interface that allows to trace contours with the mouse and identify extremal points by clicking on the image.

\spl provides plug ins to support any data source. At the moment, a data link to \sw and to the MasterLens database\needcite are implemented.

The modelling process consitis of three basic steps:
\begin{enumerate}
  \item identify lensed images and separate them from other background light sources
  \item classify and order images accoring to arrival time. (local minima, maxima, saddlepoints)
  \item fine tune the arrangement, identify addidional external point masses influencing the result
\end{enumerate}

\spl assists volunteers with several features in this process.
Step 1 by supplying several images from several bands (not yet implemented in \sw).
Step 2 by restricting the user, only valid configuration can be entered, the odd number theorem\needcite is taken care of.


Give a guess for the maximum, minimum and saddle points.  Program
tries find $\kappa(x,y)$ that reproduces these properties exactly, and
looks reasonably like a galaxy.  Solution not unique, an ensemble
generated.
SpaghettiLens is build on top of GLASS \citep{Lubini2012}, that builds and improves upon PixeLens \citep{Saha2004}.

\spl then tries to find a mass distribution $\kappa(x,y)$ that reproduces the input parameters.
Since there is no unique solution, \spl samples the solution space and produces an ensemble of solutions, as described in the paper \citep{Lubini2012}.

Step 3 by providing the visual feedback. Users can check the generated mass distribution and synthetic image.

It additinally provides plots of the modelled mass distribution and the contour lines of arrival time surface as feedback. 

It provides direct visual feedback by rendering a synthetic image side
by side the original image, such that users can directly compare the
predictions of their model to the survey image. Explanation: Synthetic
image: A rendering of the derivative of the arrival time for each
pixel. This leads to a black and white image that reasonably looks
like the visual appearace of the model.

Using this direct visual feedback, even inexperienced users can
successfully model lenses using a iterative approach if they know what
to look for.

In the next step, several volunteers can work together / exchange their ideas and models and try to improve previous modelling attempts by others.
This leads to a set of models organised in a branching, tree like structre for each model.
Volunteers then organise them selves to narrow down the different branches and come up with a few consensus models in the end.







