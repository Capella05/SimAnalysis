\begin{figure}
  \centering
  \subfigure{\includegraphics[width=0.45\textwidth]
             {fig/sims/006941/input.png}} \\
  \subfigure{\includegraphics[width=0.45\textwidth]
             {fig/sims/007022/input.png}} \\
  \subfigure{\includegraphics[width=0.45\textwidth]
             {fig/sims/006919/input.png}}
  \caption{Examples of Spaghetti input.  The models from these appear
    later in Figures \ref{fig:6941}, \ref{fig:7022} and \ref{fig:6919}.
  }
  \label{fig:input-spag}
\end{figure}

\subsection{\spl} \label{sec:SpaghettiLens}

\spl implements the ideas of the arrival-time surface into a modeling
program.

The first modeling step with \spl is for the user to call up the \sw
of interest, and make an educated guess for the topography of the
arrival-time surface --- specifically, the locations of the maxima,
minima and saddle points that correspond to images of the brightest
part of the source.  Input is given by tracing a proposal for the
saddle-point contours.  \Figref{input-spag} shows some example
inputs --- and incidentally also indicates the origin of the name
\spl.

\spl, as implemented so far, assumes that the lens is dominated by a
single galaxy, and that the centre of that galaxy is the sole maximum.
Once the maximum has been identified, we wish to characterize the rest
of the arrival-time surface in relation to it.  From geometry, any
surface that has a central maximum and goes high at the edges must
have a lima\c con-like contour, meaning a two-looped curve with one
loop inside the other.  The maximum lies within the inner loop, a
minimum lies between the two loops, and a saddle point lies at the
self-intersection of the contour.  The top panel of
\figref{input-spag} shows an example: red marks a suggested
maximum, green a saddle point, and blue a minimum.  (The small pink
dots are just help sketch the proposed contour.)  The middle and lower
panels of \figref{input-spag} show a more complex situation.
The minimum has split into three images: two new minima with a saddle
point in between.  In this situation the contour through the new
saddle point forms a figure of eight, its two loops enclosing the two
new minima.  The configuration of a lima\c con, possibly with a figure
of eight inside, suffices to give a preliminary description of the
arrival-time surface for nearly all cases where the lens is dominated
by one galaxy, and the source is a single small object.

The exact placement of the pink spaghetti loops shown in
\Figref{input-spag} has no significance.  Their utility is
just to help mark up the image as maximum, minima and saddle points.
As the figure suggests, the marking up can be done very easily with a
mouse.  Human intuition is required to (a)~identify lensed images and
separate them from other background light sources, and (b)~classify
and order images accoring to arrival time.  But the computer helps by
allowing only valid lensing configurations to be entered, and ensuring
that the odd image theorem\needcite is taken care of.

In the second modelling step, once the user has sketched a proposed
spaghetti configurations, \spl sends the input to its server-side
modelling engine, called GLASS.  The task of GLASS is to find a mass
distribution $\kappa(x,y)$ that exactly reproduces the locations of
the maximum, minima and saddle points.  Now, this criterion by itself
is extremely underdetermined --- there are infinitely many mass
distributions that will reproduce a given set of maxima, minima and
saddle points, but typically they (a)~produce lots of extra images,
and (b)~look very unlike galaxies.  Additional assumptions (a prior)
are necessary.  GLASS uses a prior based on suggestions by
\cite{1997MNRAS.292..148S}.
\begin{enumerate}
\item The mass distribution is built out of non-negative tiles of
  mass.  (Sometimes these tiles are called mass pixels, but we should
  emphasize that they are unrelated to image pixels, and are much
  larger.)
\item There is a notional lens centre, say $(x_0,y_0)$ which is
  identified with the maximum of the arrival time.  The source can
  have an arbitrary offset with respect to the lens centre.
\item The mass distribution must be centrally concentrated, in two
  respects.  First, the circularly averaged density must fall away
  like $$ \left[(x-x_0)^2+(y-y_0)^2\right]^{-1/2}$$ or more steeply.
  Second, the direction of increasing density at any $(x,y)$ can point
  at most $45^\circ$ away from from $(x_0,y_0)$.
\item The lens must symmetric with respect to $180^\circ$ rotations
  about $(x_0,y_0)$.  This symmetry assumption can be relaxed if the
  user wishes.
\end{enumerate}
There are still infinitely many models that satisfy both data and
prior constraints, but now they are more credible as galaxy lenses.
It is then possible to generate an ensemble of models.  The sampling
technique used by GLASS is described in \citep{Lubini2012}, and
improves upon earlier techniques \citep{2000AJ....119..439W,Saha2004}.
Typically, ensembles of 200 models are used.  That is to say, what we
call a \spl model is really the mean of an ensemble of 200 models, and
its estimated uncertainty is the range covered by the whole ensemble.

In the third modelling step, having generated a model ensemble, \spl
postprocesses it to present results and diagnostics to the user for
inspection.  This takes the form of three figures.
\begin{enumerate}
\item A grayscale plus contour map of the mass distribution.
\item A contour map of the arrival-time surface.
\item A synthetic image of the lensed features.
\end{enumerate}
After examining this feedback, the user can archive the results for
discussion, or modify their input and try again, or discard the
attempt altogether.

The fourth modelling step is discussion among modellers and iteration
on the model.  Any archived model can be revised by another user: they
can modify the spaghetti configuration slightly or drastically, or
change options like the size of the mass tiles.  Particularly
interesting lens candidates lead to trees of models in this way.
Forum discussion then prunes the tree, focussing attention on one or a
few models.

