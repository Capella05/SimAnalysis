\section{Introduction}

Our present picture of the formation of the largest structures in the
Universe --- galaxies and clusters of galaxies --- is that
gravitational instabilities (whose early stages are observable as
fluctuations of the microwave background) caused collapse into
gravitationally bound structures.  Dark matter would have dominated
the initial collapse, but once deeps wells in the gravitational
potential appeared, gas would have fallen into them, while radiating
away its gravitational potential energy through atomic processes, to
form the first stars and galaxies.  With time, galaxies and clusters
would continue to merge and change their form.  This rough outline is
well-established now, but many puzzles remain to be solved before our
understanding is satisfactory.  To progress towards solving these, we
need more knowledge of dark matter.  And to study dark matter,
gravitational lensing is attractive, because it depends only on mass.

There are many manifestions of gravitational lensing in astrophysics,
but the most spectacular are multiple images.  Gravitational lensing
depends on the sky-projection mass density, and this fact has a
counter-intuitive consequence: nearby galaxies do not produce multiple
images, because their projected density is not high enough, but beyond
100~Mpc or so, most galaxies are potential lenses.  Even so, for a
potential lens to really lens, a background source in the lensed
region is required.  So, in practice galaxies are very rarely observed
to lens.  The same tendency applies to clusters of galaxies, but since
clusters are bigger, they are more likely to lens.

In lensing galaxies, the observable lensing usually involves a single
object multiply imaged.  It is therefore desirable to have large
numbers of lenses, and this is the strategy researchers follow.  The
largest single study so far uses 58 lenses \citep{2009ApJ...703L..51K}.
Other studies have compared lens models in more detail with stellar
mass \citep{2011ApJ...740...97L} or used time-delay information from
lensed quasars to infer the Hubble time
\citep{2008ApJ...679...17C,2010ApJ...712.1378P}.

The trend is clear: researchers want to find many more lenses and
model them in more sophisticated ways.

A good summary of the observational situation is offered by
Figure~\ref{fig:masterlens}, which shows known secure multiply-imaging
lenses.  The non-uniformity on the sky is not intrinsic, it just
indicates the density of deep surveys up to 2013.  At HST resolution,
of order 1~square degree of sky must be searched to find a lens.
Older ground-based surveys yield a lens per roughly 10~square degree
of survey area.  The large surveys starting now
(DES\footnote{\tt http://pan-starrs.ifa.hawaii.edu} or
PanStarrs\footnote{\tt http://www.darkenergysurvey.org}) the expected
yield is something in between.  Given their unprecedented survey area,
these are likely to yield thousands of lenses.  Over the 2020s,
LSST\footnote{\tt http://www.lsst.org/} can be expected to give us ten
thousand lenses.

The question now arises: can lenses be found and modelled
automatically in large surveys?  Work using software robots
\citep{2009ApJ...694..924M} gives some interesting and unexpected
results.  In clean lensing system in uncrowded fields with high
signal-to-noise, the robots to very well.  In most situations,
however, robots miss lenses (low completeness) or contaminate the
results with non-lenses (low purity).  Robots can be made to
prioritize completeness or purity, but they cannot deliver both.

In response to the lessons learned from automated lens searches, the
{\em Spacewarps\/} project\footnote{\tt http://www.spacewarps.org} has
(Saha is a co-investigator) has been launched.  {\em Spacewarps\/} is
part of the {\em Zooniverse\/} family of citizen-science
projects,\footnote{\tt http://www.zooniverse.org} where members of the
public are invited to analyze different kinds of scientific data which
are too difficult for robots and too large for specialists.  In {\em
  Spacewarps\/} itself, since the launch in May 2013, $\sim100$ square
degrees (or a quarter of a percent of the sky) has been examined.  The
survey is divided into patches of $400\times 400$ pixels, each of
which is seen by ten volunteers.  Simulated lenses are mixed in with
the data, both to help train volunteers on what to look for, and to
estimate completeness and purity.  A list of candidates is being
processed, but intermediate results show some very good candidates, as
well re-discoveries of a few known lenses.

These encouraging results now raise the question: could modelling of
the lenses also be done by volunteers?  Modelling is a much much more
difficult task than searching for lens candidates, as it requires some
expert knowledge.  Nonetheless, a subset of the {\em Spacewarps\/}
volunteers are quite experienced from earlier projects, having spent a
thousand hours or more with data.  Some of these experienced
volunteers are very interested in more demanding projects.  This also
is a general trend \citep[cf.][]{Khatib22112011}.  The present project
then suggests itself.

The purpose of this study was to provide a means to model a large number of gravitational lenses by showing that gravitational lens modeling can be learned / done by volunteers.
We suggest, that volunteers will be as successful as professionals with modeling if provided with an easy to use tool with visual feedback (What you see is what you get, WYSIWYG) and a minimal set of instructions.
Volunteers will then crowd work (using buzz word here ;) ) / work collaborative on modeling lenses from several sources / groups at a central place.
Since this is an iterative learning process, the more involved volunteers will quickly gain knowledge that can be passed down to new volunteers.
That creates a social structure that scales well with the number of volunteers, as other projects\needcite have already shown.
Finding people working as volunteers has been shown to be successful last but not least by \sw and the whole galaxy zoo project.
To test the people's abilities, we investigated the performance of a few volunteers modeling a set of simulated lenses.
We tested the ability to correctly identify lensed images and reproduce similar mass map of the lens.


