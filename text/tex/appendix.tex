\appendix

\section{More on Lensing Theory} \label{sec:more-theory}

In \secref{Fermat}, for the sake of a more intuitive
explanation, we suppressed some constant factors in
equations \eqref{eq:Ageom}, \eqref{eq:Aarriv} and \eqref{eq:Poisson}.
Here we fill in the details.

First let us recall the formulas for comoving distances.
\begin{equation}
\int \frac{dz}{\sqrt{\Omega_m(1+z)^3 + \Omega_\Lambda}}
\end{equation}
\begin{equation}
\begin{aligned}
&D_S    &                                &\int_0^{a_S} \\
\noalign{\medskip}
&D_{LS} & = \frac c{H_0} \ \times \quad  &\int_{z_L}^{z_S} \\
\noalign{\medskip}
&D_L    &                                &\int_0^{z_L}
\end{aligned}
\end{equation}
Comoving distances $D$ with angular-diameter distances $d$, as follows.
\begin{equation}
\begin{aligned}
D_S &= (1+z_S) \, d_S \\
D_{LS} &= \frac{1+z_S}{1+z_L} \, d_{LS} \\
D_L &= (1+z_L) \, d_L
\end{aligned}
\end{equation}
Locations on the lens plane can be replaced angular coordinates on the
sky, as
\begin{equation}
(x,y) = d_L (\theta_x,\theta_y) \,.
\end{equation}

The $A$ times and the $\kappa$ density are related to physical arrival
time $t$ and density $\Sigma$ as
\begin{equation}
\begin{aligned}
A           &= \frac{cD_L}{(1+z_L)^2} \frac{D_{LS}}{D_S} \times t \\
\kappa(x,y) &= \frac{4\pi G}{c^2} \frac{D_L}{1+z_L} \frac{D_{LS}}{D_S}
               \times \Sigma(x,y)
\end{aligned}
\end{equation}
Letting the source be behind $(s_x,s_y)$ rather than behind the
origin, we have
\begin{equation}
\begin{aligned}
t_{\rm geom} &= \frac{(1+z_L)^2}{2cD_L} \frac{D_S}{D_{LS}}
\left( (x-s_x)^2 + (y-s_y)^2 \right)
\nabla^2 t_{\rm grav} &= -(1+z_L)\frac{8\pi G}{c^3} \, \Sigma(x,y)
\end{aligned}
\end{equation}
We can now compare with equations (2.1) to (2.6)
from \cite{1986ApJ...310..568B}.

