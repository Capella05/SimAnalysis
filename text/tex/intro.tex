\section{Introduction}


Gravitational lenses (GLs), predicted by ART, can be used as a tool by astronomers to examine many properties of the cosmos.
They allow for example to estimate the masses and their distribution of galaxies, automatically including the hard to grasp dark matter.
Further more, they allow an estimate on cosmological parameters like the Hubble constant \citep{Saha2006} and the mass density \needcite.

%To accheve this, one needs to find / identify lenses and in a second step model those lenses to get involved parameters.

How to find gravitational lenses? Huge amount of image data from surveys that needs to be processed.
Even more comming in the future with new surveys \needcite
Robotic procession has been suggested \needcite, and tested \needcite, but has failed so far to be convincing \needcite.
Another suggestion involves humans, volountiers \footnote{A volounteer is considered everybody that has no background in astro physics.}.
\sw uses this approach with great succes so far.\needcite


Next step involves modeling. that needs advanced knolegde and takes a lot of time.
Too much time to be done by astronomers them selves as the results from new surveys and identifiers like \sw come in.
So there is a demand of a means to modell a great amount of identified lens data, that scales with increasing data amount.


The purpose of this study was to provide a means to model a large amount of gravitational lenses by showing that gravitational lens modelling can be learned / done by volunteers.
We suggest, that volunteers will be as successful as professionals with modelling if provided with an easy to use tool with visual feedback (WYSIWYG) and a minimal set of instructions.
Voulnteers will then croud work (using buzz word here ;) ) / work collarobartive on modelling lenses from several sources / groups at a central place.
Since this is a iterative learning process, the more involved volunteers will quickly gain knoledge that can be passed down to new volunteers.
That creates a social structre that scales well with the number of volunteers, as other projects\needcite have already shown.
Finding people working as volunteers has been shown to be successful lst but not least by \sw and the whole galaxy zoo project.
To test the peoples abilities, we invesigated the performance of a first set of volunteers modelling a set of simulated lenses.
We tested the ability to correcly identify lensed images and reproduce simmilar mass maps of the lens.
