\section{Outlook} \label{sec:todo}

The lens-modelling challenge indicates that the Spacewarps collective
is good, not only for identifying lens candidates for follow-up, but
modelling candidates as well.

There is, however, plenty of room for improvement.

First, the particular modelling strategy implemented is not the only
one possible.  \spl requires modellers to characterise the overall
image structure in abstract terms based on Fermat's principle, and the
placement of a mass distributions is done by the computer.  In other
modelling tools, the user puts down a trial mass distribution and has
the machine refine it.  A few of these modelling programs have also
been designed with citizen science in mind, and would be interesting
the in Spacewarps environment.

Second, \spl needs some enhancements.

One, model the source as well.


Another, lenses with more than one maximum can be allowed for.  See
Figure 5c in \citep{2001ApJ...557..594R} and Figure 4b in
\cite{2003ApJ...590...39K}.

Other modelling approaches also possible.

As seen in \secref{tests.t1}, volunteers fail mostly in two situations:
when to identify an arc like structure while placing the points
and in 5 image configurations, that are quite symmetric, then to identify the correct ordering of the points.

The first can be improved by additional images from different filters (?)
Will be implemented with a next major \spl upgrade.

The second can be made less by better training (introduce the ruler), may be more filters help too?

We hope that those problems are also taken care of when staring collaborative modeling. (Already in progress)
Basics are available ('revise functionality') but better community tools would help (having overview, comparison between different models etc...)



