\section{Outlook} \label{sec:todo}

The lens-modeling challenge indicates that the Spacewarps collective
is good, not only for identifying lens candidates for follow-up, but
modeling candidates as well.

There is, however, plenty of room for improvement.

First, the particular modeling strategy implemented is not the only
one possible.  \spl requires modelers to characterize the overall
image structure in abstract terms based on Fermat's principle, and the
placement of mass distributions is done by the computer.  In other
modeling tools, the user puts down a trial mass distribution and has
the machine refine it.  A few of these modeling programs have also
been designed with citizen science in mind, and would be interesting
in the Spacewarps environment.

Second, \spl needs some enhancements.

\begin{itemize}
\item Currently, \spl does not attempt to model the source shape; the
  user identifies the brightest points on the image, and these are
  taken as images of a point-like source, whose positions must be
  reproduced exactly. For generating a synthetic image, a conical
  source profile is assumed. Fitting for the source profile to
  optimize resemblance to the observed lensed image after the lens
  model has been generated, is algorithmically straightforward and
  planned to be implemented.  This would alleviate another problem
  with \spl, which is that there is no numerical figure of merit, and
  assessment of a model is a judgment call based on the synthetic
  image, and on whether the mass distribution and the arrival-time
  surface show suspect features.
\item Another current limitation in \spl is that the lens is assumed
  to be dominated by one galaxy, which puts most galaxy-group lenses
  beyond the reach of the modeler. Since complicated group lenses
  are some of the most interesting candidates present, removing this
  limitation is most desirable.  From the users' point of view, it
  would mean that spaghetti contours with more than one maximum can be
  allowed.  For examples, see Figure 5c in \citep{2001ApJ...557..594R}
  and Figure 4b in \cite{2003ApJ...590...39K}.
\item Currently, a single false-color composite is used as the data.
  An option could be added to use all available filters, individually
  or in combination, at the user wishes.
\end{itemize}

The third desirable avenue of improvement is to facilitate
collaborative work.

\begin{itemize}
\item As mentioned above, the option of revising an already-archived
  model is already available.  Desired now are tools for comparing
  different models of a given system, both visually and through
  different statistical measures.  As evidenced by a current
  collaborative modeling effort, a particularly interesting candidate
  can lead to an extended discussion and dozens of models.
\item Better tutorial materials are also needed, and this would
  address some of the problem areas found in the modeling challenge.
  For example, we saw in \secref{tests.t1} that volunteers are most
  prone to making errors in two situations: when in identifying an arc-
  like structure while placing the points, and in identifying the
  correct ordering of the points in nearly-symmetric configurations.
\end{itemize}

